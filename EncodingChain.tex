\chapter{Encoding FEC Chain} \label{chap:encoder}

% In this document explain all the things which I have done until the midterm presentation.
% All the details, Optimization techniques I have employed.
% Following text might be useful for writing the text.
% Also my midterm presentations/office presentations will be really useful for writing this part.  
% Reference paper for explaining the different components of the this FEC chain "Design of Polar codes for 5G NR radio"

%Work I have done Until now.
%
%Optimizations to the original implementations until now.
%Generic optimizations
%- Using optimization primitives such as likely and unlikely.
%- Aligning memory to 32 bytes so copying of data can be vectorized.
%Polar transform optimization	
%- Replace binary additions with xor. instead of addition and then modulus two.
%- division and multiplications by left and right shift operations.
%- Avoided copy operations in polarTransform operations.
%Optimization in getting reliability indices.
%- Avoided remove and erase operations which have huge overhead. Wrote a efficient mechanism(reduced the latency by 176 us).
%- Instead removing and erasing I mark the element as removed.
%- Since the reliability indexes won't change. I built a look up table in place of searching all (1024)indices reduced the latency by 40us
%- Avoided copying operations of interleaved indexes.
%- Unrolled the loop to reduce the jumps.
%Rate matching optimizations.
%- optimization in subblock interleaving, Rewrote the logic to avoid E number of division and modulus operations.
%- Unrolled the for loops in subblock inteleaving method.
%- Implemented optimal version of bit selection, Avoided E number of modulus operations which are very costly.
%- Again optimization primitives for helping the branch predictor.
%
%
%Fast version of Encoding API's.
%In the original implementation of the polar encoding each of the bit is treated as 32 bit integer. This is highly inefficient
%when the goal is to process multiple bits at time. With each bit considered as 32 bit integer SIMD instructions won't provide
%any performance improvement. Reason is SIMD instruction can process multiple bits at time. avx2 instructions 256bits at a time.
%if we have 32 bits to represent a single bit. we can process only 8 bits at time. Which doesn't significantly improve the
%performance. To avoid this disadvantage and make use of SIMD capability. each 64 bit integer is considered as 64 bits of data.
%so one avx2 instruction can process 256 data bits in a single instruction.
%- Built a look up table to avoid last eight stages of polar encoding instead of traversing till end of tree.
%- Implemented SIMD instruction based encoding. Encoding happens within 0.6 us for N = 512.
%- Implemented optimal version of CRC calculation which can calculate CRC for PDCCH chain within 0.8 us. Original implementation was taking 7 us.
%- Implemented a bit interleaver which can deal with this format of data.

In this section, complete polar encoding FEC chain used in 5G is explained. Methods used for code profiling and latency measurement are presented. After figuring out the latency contributors code optimization/algorithm optimizations employed during the FEC chain development are presented and then presents how SIMD feature of the modern processors is exploited to obtain the low latency for PBCH and PDCCH FEC chains and frozen set selection algorithm improvement with the aid of look up table, finally presents the encoding process as traversal of a binary tree and how the encoding latency can be improved by pruning the tree hence avoiding the tree traversal instead using a lookup table to obtain the encoded result.

In 5G framework, Polar codes are used in downlink to encode downlink control information (DCI) over physical downlink control channel (PDCCH) and for payload in physical broadcast channel (PBCH). In uplink, to encode uplink control information (UCI) over the physical uplink control channel (PUCCH) and the physical uplink shared channel (PUSCH). In this work, notations introduced in 3GPP technical specification\cite{3gpp.38.212} are used.

The figure ~\ref{fig:5g_fec_chain} represents the complete polar FEC chain for PBCH and PDCCH in downlink. Let's look at each of the components briefly to understand the FEC chain. In general $A$ bits have to be transmitted over a code of length $E$ code bits. $L$ CRC bits are added to the information bits, resulting in  $K = (A + L)$ bits. These $K$ bits are passed through an interleaver. Interleaved bits are concatenated with a parity bits and assigned to information set to obtain a vector $\boldsymbol{u}$. Encoding is done with a mother code with parameters $(N,K)$, with $N = 2_{n}$. Encoding is performed $\boldsymbol{d = uG_{N}}$ the generator matrix $\boldsymbol{G_{N} = G^{\otimes n}}$ obtained by $n^{th}$ kronecker product of Arikan matrix. Encoded codeword $\boldsymbol{d}$ passed through a subblock interleaver which divides the codeword in to blocks of 32 bits and performs interleaving between them according to 32 integers (interleaving pattern is nothing but a bit reversal of bit position) shown in figure. \textbf{add picture of bit reversal operation}. After subblock interleaving is completed rate matching is carried out. To map $N$ to $E$ bits. Rate matching can repetition, puncturing or shortening. This decision is taken based on the value of $E$, $N$ and $K$. Finally to improve the error correction performance channel interleaving is done. This section of the report presents  implementation details of each of these operations in an algorithmic level with small code snippets whenever necessary. Analyzes latency introduced by different sections of FEC chain and also presents the algorithmic and platform specific optimizations.

\begin{figure}[h]
	\centering
	\includegraphics[width=0.7\textwidth]{./figures/5GFECChain.pdf}
	\caption{Polar Encoding FEC chain for PDCCH/PBCH}
	\label{fig:5g_fec_chain}
\end{figure}

\section{Data packing and Unpacking Operations}
Typically in software implementations, for clarity and ease implementation each bit of information is represented with 32-bit or 64-bit integers. Due to the presence of only one bit of information in each integer if want to encode/decode 1024 bits, then 1024 integers are involved in encoding/decoding process. However this isn't the case in hardware implementations since each bit can be processes in Harware Description languages (HDL). Representing each one bit of information using 32/64-bit integer has following disadvantages. 

\begin{description}[font=$\bullet$~\normalfont]
	\item Increased memory requirement: If represent each information bit using 64-bit integer
	\item Results in more cache misses:
	\item Serializes encoding/decoding:
\end{description}

$\bullet$ Requires more memory for storing the data. \newline
$\bullet$ Results in more cache misses due more memory accesses. \newline
$\bullet$ Serializes encoding and decoding procedures. \newline


Namely it requires more memory and inherently serializes the process of encoding/decoding i.e each individual bit is processed sequentially.]

For example 



Explain why and how packing and unpacking is carried out and the advantages of data packing and unpacking.

\section{CRC calculation}
Explain why CRC attachment is considered in whole FEC chain, how it is useful for decoding polar codes. Explain algorithmic complexity of CRC calculation, How much time it was taking, How the optimization is carried out reduce the CRC calculation time.

\section{Input Bit Interleaver}
Any optimization carried out in here, need to be explained clearly, Why this input bit interleaver is necessary. How much time this function takes.

\section{Polar code construction}
Here explain the algorithm how frozen/information/parity indices are selected. With an algorithm and flow chart which makes very clear/easy to understand the algorithm. Explain the effect of puncturing and shortening on the bit reliability. May be mode should be selected, either as puncturing/shortening at constructor. And corresponding operations are performed accordingly. Give also the details about information bit insertion reliable locations. Explain the effect of puncturing and shortening on the reliability of bit channels.

\section{Polar Encoding}
Explain how the matrix multiplication is transformed into recursive formulation. Represent the encoding procedure as traversing through binary tree. How the parallelism of SIMD processor is exploited speed up the encoding process. Explain the employed tree pruning method.

\section{Results Comparison}



