\chapter{Encoding FEC Chain} \label{chap:encoder}

% In this document explain all the things which I have done until the midterm presentation.
% All the details, Optimization techniques I have employed.
% Following text might be useful for writing the text.

Work I have done Until now.

Optimizations to the original implementations until now.
Generic optimizations
- Using optimization primitives such as likely and unlikely.
- Aligning memory to 32 bytes so copying of data can be vectorized.
Polar transform optimization	
- Replace binary additions with xor. instead of addition and then modulus two.
- division and multiplications by left and right shift operations.
- Avoided copy operations in polarTransform operations.
Optimization in getting reliability indices.
- Avoided remove and erase operations which have huge overhead. Wrote a efficient mechanism(reduced the latency by 176 us).
- Instead removing and erasing I mark the element as removed.
- Since the reliability indexes won't change. I built a look up table in place of searching all (1024)indices reduced the latency by 40us
- Avoided copying operations of interleaved indexes.
- Unrolled the loop to reduce the jumps.
Rate matching optimizations.
- optimization in subblock interleaving, Rewrote the logic to avoid E number of division and modulus operations.
- Unrolled the for loops in subblock inteleaving method.
- Implemented optimal version of bit selection, Avoided E number of modulus operations which are very costly.
- Again optimization primitives for helping the branch predictor.


Fast version of Encoding API's.
In the original implementation of the polar encoding each of the bit is treated as 32 bit integer. This is highly inefficient
when the goal is to process multiple bits at time. With each bit considered as 32 bit integer SIMD instructions won't provide
any performance improvement. Reason is SIMD instruction can process multiple bits at time. avx2 instructions 256bits at a time.
if we have 32 bits to represent a single bit. we can process only 8 bits at time. Which doesn't significantly improve the
performance. To avoid this disadvantage and make use of SIMD capability. each 64 bit integer is considered as 64 bits of data.
so one avx2 instruction can process 256 data bits in a single instruction.
- Built a look up table to avoid last eight stages of polar encoding instead of traversing till end of tree.
- Implemented SIMD instruction based encoding. Encoding happens within 0.6 us for N = 512.
- Implemented optimal version of CRC calculation which can calculate CRC for PDCCH chain within 0.8 us. Original implementation was taking 7 us.
- Implemented a bit interleaver which can deal with this format of data.

\begin{figure}[!htb]
	\centering
	\includegraphics[width=0.8\textwidth]{./figures/5GFECChain.pdf}
	\caption{Polar Encoding FEC chain for PDCCH/PBCH}
	\label{fig:5g_fec_chain}
\end{figure}